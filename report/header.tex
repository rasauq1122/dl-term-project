% !TEX program = xelatex
% !TEX root = main.tex
\documentclass[11pt]{article}

\usepackage{amsmath}
\usepackage{fontspec}
\usepackage{graphicx}
\usepackage{import}
\usepackage{kotex}
\PassOptionsToPackage{table}{xcolor}
\usepackage{calc}
\usepackage{listings}
\usepackage{indentfirst}
\usepackage{tabularx}
\usepackage{ulem}
\usepackage{multicol}
\usepackage{epigraph}
\usepackage[many]{tcolorbox}
\usepackage{geometry}
\usepackage{titletoc} % titletoc 패키지 사용
\usepackage{xcolor}
\usepackage{minted}
% \usepackage{jupynotex}

\linespread{1.2}
\everymath{\displaystyle}
\geometry{margin=1in}

\graphicspath{ {./images/} }
\lstset{basicstyle=\footnotesize\ttfamily,breaklines=true}

\newcommand{\translation}[1]{\textsuperscript{#1}}

\setlength\fboxsep{0pt}

\newcommand{\complexity}[1]{$\mathcal{O}\left({#1}\right)$}
\newcommand{\difficulty}[1]{\includegraphics[width=1em,natwidth=1000,natheight=1000]{#1.svg.png}}
\newcommand{\norm}[1]{\left\lVert#1\right\rVert}

\lstdefinestyle{mystyle}{
    backgroundcolor=\color{gray!10},
    basicstyle=\ttfamily\footnotesize,
    commentstyle=\color{green!40!black},
    keywordstyle=\color{blue},
    numberstyle=\tiny\color{gray},
    numbers=left,
    stringstyle=\color{red},
    breakatwhitespace=false,
    breaklines=true,
    captionpos=b,
    keepspaces=true,
    showspaces=false,
    showstringspaces=false,
    showtabs=false,
    tabsize=4
}

\lstset{style=mystyle}